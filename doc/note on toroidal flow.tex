\documentclass[12pt]{article}
\usepackage{graphicx}
\usepackage{cite}
\usepackage{CJK}
\usepackage{amsmath}
\usepackage[colorlinks,linkcolor=black,anchorcolor=blue,citecolor=blue,urlcolor=red]{hyperref}
\usepackage{parskip}
\usepackage{color}
\setlength{\parindent}{0cm}

\begin{document}
\begin{CJK*}{UTF8}{gbsn}
\title{\textbf{Adding toroidal flow in GEM for adiabatic electron model}}
\author{汤炜康}
\date{}
\maketitle
\tableofcontents
\newpage

\section{Guiding center drift due to toroidal flow}
There are three terms needed to be added in the drift velocity of the guiding center $V_{G}$, 
according to Eq. 21 in \cite{sugama98}, i. e.  

\begin{equation}
    \frac{cm_a}{e_aB}\hat{b} \times (\mathbf{U} \cdot \nabla \mathbf{U}), \label{eq1}
\end{equation}
\begin{equation}
    \frac{cm_av_{\parallel}^{'}}{e_aB} \hat{b} \times (\hat{b} \cdot \nabla \mathbf{U}), \label{eq2}
\end{equation}
\begin{equation}
    \frac{cm_av_{\parallel}^{'}}{e_aB} \hat{b} \times (\nabla \mathbf{U} \cdot \hat{b}). \label{eq3}
\end{equation}

In the right-handed coordinate system (R,Z,$\zeta$), considering the toroidal flow with the form of $\mathbf{U}(\mathbf{R}) = -\omega_0(\psi_p)R^2 \nabla \zeta$,
\begin{equation}
    \mathbf{U} \cdot \nabla \mathbf{U} = - \frac{U^2}{R} \hat{R}
\end{equation}
\begin{equation}
    \begin{split}
        \hat{b} \times (\mathbf{U} \cdot \nabla \mathbf{U}) &= \frac{U^2}{R} \hat{R} \times \hat{b}\\
        &= \frac{U^2}{RB} \hat{R} \times (\frac{f}{R}\hat{\zeta} + \frac{\psi_{p}^{'}(r)}{R}(\frac{\partial r}{\partial R}\hat{Z} - \frac{\partial r}{\partial Z}\hat{R}))\\
        &= \frac{U^2}{RB} (-\frac{f}{R}\hat{Z} + \frac{\psi_{p}^{'}(r)}{R} \frac{\partial r}{\partial R}\hat{\zeta}) \label{eq5}
    \end{split}
\end{equation}

Then Eq. \ref{eq1} becomes  
\begin{equation}
    \frac{cm_aU^2}{e_aB^2R^2}(-f\hat{Z} + \psi_{p}^{'}(r)\frac{\partial r}{\partial R} \hat{\zeta}).
\end{equation}

Eq. 6 $\cdot \nabla x$ is
\begin{equation}
    \frac{cm_aU^2}{e_aB^2R^2} \cdot (-f) \frac{\partial r}{\partial Z}.
\end{equation}

This term can be coded as \texttt{-Up**2*fp*srbzp/bfldp**2/radiusp**2}, where \texttt{U(0:nr,0:ntheta)} is a new variable needed to be declared, representing the equlibrium toroidal flow. 
Actually, we only need to declare a new variable \texttt{omega(0:nr)}, \texttt{U(0:nr,0:ntheta)=-omega(0:nr)*radius(0:nr,0:ntheta)}.

For Eq. 6 $\cdot \nabla y$, 
\begin{equation}
    y = \frac{r_0}{q_0}\int_{0}^{\theta}\hat{q}(r,\theta^{'})d\theta^{'}-\zeta)=\frac{r_0}{q_0}(q\theta_f-\zeta)
\end{equation}
\begin{equation}
    \nabla y = \frac{\partial y}{\partial r} \nabla r + \frac{r_0}{q_0} \hat{q} \nabla \theta - \frac{r_0}{q_0} \nabla \zeta,\label{eqdely}
\end{equation}
where $\hat{q} = q \frac{\partial \theta_f}{\partial \theta}$. 
\begin{equation}
\begin{split}
    \textrm{Eq.}\ 6 \cdot \nabla y &= \frac{cm_aU^2}{e_aB^2R^2}(-f\hat{Z} + \psi_p^{'}(r)\frac{\partial r}{\partial R} \hat{\zeta}) \cdot \nabla y \\
                                   &= \frac{cm_aU^2}{e_aB^2R^2}(-f \frac{\partial y}{\partial r} \frac{\partial r}{\partial Z} - \frac{r_0}{q_0} \hat{q} f 
                                      \frac{\partial \theta}{\partial Z} - \frac{r_0}{q_0R} \psi_p^{'}(r)\frac{\partial r}{\partial R}), 
\end{split}
\end{equation}
i. e. \texttt{-Up**2/bfldp**2/radiusp**2 * (fp*dydrp*srbzp + r0/q0*qhatp*fp\\ *thbzp + r0/q0/radiusp*psipp*srbrp).}

So far, all the terms about Eq. \ref{eq1} is finished. Let's start with Eqs. \ref{eq2} and \ref{eq3}.
Still in the $(R,Z,\zeta)$ coordinate, one could find Eq. \ref{eq2} is identical to Eq. \ref{eq3} as
\begin{equation}
    \mathbf{U}\cdot\nabla\hat{b} = \hat{b}\cdot\nabla\mathbf{U} = - \frac{Uf}{R^2B}\hat{R} - \frac{U}{R^2B}\psi_p^{'}(r)\frac{\partial r}{\partial Z}\hat{\zeta}.
\end{equation}
To derive Eq. 2 or Eq. 3, 
\begin{equation}
\begin{split}
    &\ \ \ \  \frac{cm_av_{\parallel}^{'}}{e_aB} \hat{b} \times (\mathbf{U}\cdot\nabla\hat{b}) \\&= \frac{cm_av_{\parallel}^{'}}{e_aB}
    [-\frac{Uf^2}{B^2R^3}\hat{Z} + \frac{Uf\psi^{'}_p(r)}{B^2R^3}\frac{\partial r}{\partial R}\hat{\zeta}\\
     &\ \ \ \ -\frac{U}{B^2R^3}\psi_p^{'2}(r)\frac{\partial r}{\partial R}\frac{\partial r}{\partial Z}\hat{R}
     -\frac{U}{B^2R^3}\psi_p^{'2}(r)(\frac{\partial r}{\partial Z})^2\hat{Z}]\\
    &=\frac{cm_av_{\parallel}{'}U}{e_aB^3R^3}[-\psi_p^{'2}(r)\frac{\partial r}{\partial R}\frac{\partial r}{\partial Z}\hat{R}
     -(f^2 + \psi_p^{'2}(r)(\frac{\partial r}{\partial Z})^2)\hat{Z} + f\psi_p^{'}(r)\frac{\partial r}{\partial R}\hat{\zeta}].\label{eq14}
\end{split}
\end{equation}
Then, 
\begin{equation}
\begin{split}
    \textrm{Eq.}\ \ref{eq14} \cdot\nabla x &= -\frac{cm_av_{\parallel}{'}U}{e_aB^3R^3}[\psi_p^{'2}(r)(\frac{\partial r}{\partial R})^2\frac{\partial r}{\partial Z}
    +(f^2 + \psi_p^{'2}(r)(\frac{\partial r}{\partial Z})^2)\frac{\partial r}{\partial Z}]\\
    &= -\frac{cm_av_{\parallel}{'}U}{e_aB^3R^3}[\psi_p^{'2}(r)(\frac{\partial r}{\partial R})^2
       +f^2 + \psi_p^{'2}(r)(\frac{\partial r}{\partial Z})^2]\frac{\partial r}{\partial Z},
\end{split}
\end{equation}
which can be coded as \texttt{-Up*vpar/bfldp**3/radiusp**3*(psipp**2*srbrp**2 + fp**2 + psipp**2*srbzp**2)*srbzp}.

Using Eq. \ref{eqdely},
\begin{equation}
\begin{split}
    \textrm{Eq.}\ \ref{eq14} \cdot\nabla y =& -\frac{cm_av_{\parallel}{'}U}{e_aB^3R^3}[\psi_p^{'2}(r)\frac{\partial r}{\partial R}\frac{\partial r}{\partial Z}
    (\frac{\partial y}{\partial r}\frac{\partial r}{\partial R} + \frac{r_0}{q_0}\hat{q}\frac{\partial \theta}{\partial R})\\
    &+ (f^2 + \psi_p^{'2}(r)(\frac{\partial r}{\partial Z})^2)(\frac{\partial y}{\partial r}\frac{\partial r}{\partial Z} + 
    \frac{r_0}{q_0}\hat{q}\frac{\partial \theta}{\partial Z})\\
    &+ \frac{r_0}{q_0R}f\psi_p^{'}(r)\frac{\partial r}{\partial R}].
\end{split}
\end{equation}
That is, \texttt{-Up*vpar/bfldp**3/radiusp**3*(psipp**2*srbrp*srbzp*(dydrp*srbrp + r0/q0*qhatp*thbrp)
 + (fp**2 + psipp**2*srbzp**2)*(dydrp*srbzp + r0/q0*qhatp*thbzp) + r0/q0/radiusp*fp*psipp*srbrp)}.

\newpage
\section{Parallel acceleration due to toroidal flow}
An auxiliary guiding center variable $v_{\parallel}{''}$ is defined according to 
\begin{equation}
    \varepsilon = \mu\mathbf{B_\mathrm{0}(R)} + \frac{1}{2}mv_{\parallel}^{''2} - \frac{1}{2}mU^2\mathbf{(R)} + q\Phi_1\mathbf{(R)}
\end{equation}
Notice that $v_{\parallel}{''}$ depends on ($\mathbf{R},\varepsilon,\mu$) but not $\gamma$, and $v_{\parallel}{''}=v_{\parallel}{'}+\mathcal{O}(\delta)$.
The new parallel velocity is defined with $\varepsilon^{\mu}$. Since $d\varepsilon^{\mu}/dt=\mathcal{O}(\delta^2)$,
\begin{equation}
\begin{split}
    mv_{\parallel}''\frac{dv_{\parallel}''}{dt} &= -\mu\frac{dB\mathbf{(R)}}{dt} + m\frac{dU^2}{dt} - q\frac{d\Phi_1}{dt}\\
    &=-\mu v_{\parallel}''\mathbf{b}\cdot\nabla B + mv_{\parallel}''\mathbf{b}\cdot\nabla U^2 - qv_{\parallel}''
    \mathbf{b}\cdot\nabla\Phi_1 + \mathcal{O}(\delta^2)
\end{split}
\end{equation}
We will neglect the $\mathcal{O}(\delta^2)$ terms, which include the parallel nonlinearity.
\begin{equation}
    \frac{dv_{\parallel}''}{dt} = -\frac{\mu}{m}\mathbf{b}\cdot\nabla B + \mathbf{b}\cdot\nabla U^2 - \frac{q}{m}\mathbf{b}\cdot\nabla\Phi_1 + \mathcal{O}(\delta^2)
\end{equation}
For any scalar $s(r,\theta,\zeta)$,
\begin{equation}
    \mathbf{b}\cdot\nabla s=\frac{1}{B}\frac{\psi_p'}{R}\frac{\partial s}{\partial \theta}\hat{\zeta}\cdot\nabla r \times \nabla\theta
    + \frac{f}{BR^2}\frac{\partial s}{\partial \zeta}.
\end{equation}
{\color{cyan}
\begin{equation*}
\begin{split}
    \mathbf{b}\cdot\nabla s&=\frac{1}{B}(\nabla\zeta\times\nabla\psi_p+\frac{f}{R}\hat{\zeta})\cdot(\frac{\partial s}{\partial r}\nabla r
    +\frac{\partial s}{\partial \theta}\nabla \theta+\frac{\partial s}{\partial \zeta}\nabla \zeta)\\
    &=\frac{\psi_p'}{B}(\nabla\zeta\times\nabla r)\cdot(\frac{\partial s}{\partial r}\nabla r
    +\frac{\partial s}{\partial \theta}\nabla \theta+\frac{\partial s}{\partial \zeta}\nabla \zeta)+\frac{f}{BR^2}\frac{\partial s}{\partial\zeta}\\
    &=\frac{1}{B}\frac{\psi_p'}{R}\frac{\partial s}{\partial \theta}\hat{\zeta}\cdot\nabla r \times \nabla\theta
    + \frac{f}{BR^2}\frac{\partial s}{\partial \zeta}
\end{split}
\end{equation*}
}

So,
\begin{equation}
    \mathbf{b}\cdot\nabla B=\frac{1}{B}\frac{\psi_p'}{R}\frac{\partial B}{\partial \theta}\hat{\zeta}\cdot\nabla r \times \nabla\theta
\end{equation}
\begin{equation}
    \mathbf{b}\cdot\nabla U^2=\frac{2U}{B}\frac{\psi_p'}{R}\frac{\partial U}{\partial \theta}\hat{\zeta}\cdot\nabla r \times \nabla\theta,
\end{equation}
\begin{equation}
    \mathbf{b}\cdot\nabla \Phi_1=\frac{1}{B}\frac{\psi_p'}{R}\frac{\partial \Phi_1}{\partial \theta}\hat{\zeta}\cdot\nabla r \times \nabla\theta
\end{equation}
with
\begin{equation}
    \hat{\zeta}\cdot\nabla r \times \nabla\theta = |\nabla r \times \nabla\theta|.
\end{equation}
The $\partial U/\partial\theta$ and $\partial \Phi_1/\partial\theta$ will be calculated in the \texttt{gem\_equil.f90}. Others are all existing variables in GEM. 
$\Phi_1$ is the electric potential determined by the charge-neutrality, with $\mathbf{E_n}=-\nabla\Phi_1$. For a plasma with a single ion species with ion 
temperature $T_i$ and electron temperature $T_e$, 
\begin{equation}
    e\Phi_1=\frac{m_i\omega_0^2}{2(1+T_i/T_e)}(R^2 - \langle R^2\rangle).
\end{equation}
Attention, here the bracket $\langle\cdots\rangle$ stands for the flux surface average.
\begin{equation}
    \langle R^2\rangle = \frac{1}{2\pi}\int_{-\pi}^{\pi}R^2(r,\theta)d\theta
\end{equation}
Thus,
\begin{equation}
    \partial_{\theta}(R^2 - \langle R^2\rangle) = 2R\partial_{\theta}R-\frac{R^2}{2\pi}
\end{equation}
% Considering
% \begin{equation}
% \begin{split}
%     \mathbf{B}&=\nabla\psi\times\nabla(q\theta_f-\zeta)\\
%     &=\frac{q_0}{r_0}\frac{d\psi}{dx}\nabla x\times\nabla y\\
%     &=C(x)\nabla x\times\nabla y
% \end{split}
% \end{equation}
% and
% \begin{equation}
%     \mathbf{b}\cdot\nabla x\times\nabla y=\frac{\nabla x\times\nabla y}{|\nabla x\times\nabla y|}\cdot\nabla x\times\nabla y=|\nabla x\times\nabla y|,
% \end{equation}
% \begin{equation}
% \begin{split}
%     \mathbf{b}\cdot\nabla\Phi_1&=\frac{\nabla x\times\nabla y}{|\nabla x\times\nabla y|}\cdot\nabla z\frac{\partial \Phi_1}{\partial z}\\
%     &=\frac{1}{J|\nabla x\times\nabla y|}\frac{\partial \Phi_1}{\partial z}\\
%     &=\frac{-ez}{jac*bdgxcgy}
% \end{split}
% \end{equation}

\newpage
\section{Changes in weight equation} 
Let $f=f_0+\delta f$, the perturbed part of distribution function
\begin{equation}
    \delta f = -q(\phi - \mathbf{U\cdot A})\frac{f_0}{T} + h
\end{equation}
Here, $h$ is the non-adiabatic part of $\delta f$. 

Write $h=h + h_1 + h_2 \cdots$
\begin{equation}
    h_2 = \delta f + \frac{q}{T}f_0 [(\phi - \mathbf{U\cdot A}) - \langle(\phi - \mathbf{U\cdot A})\rangle + \langle\mathbf{v'\cdot A}\rangle]
\end{equation}
\begin{equation}
\begin{split}
    \ \ \ \ &\frac{\partial h_2}{\partial t} + \langle\mathbf{\dot{R}}\rangle\cdot\nabla h_2 + \langle\dot{\varepsilon}^u\rangle\frac{\partial h_1}{\partial\varepsilon^u}
    - \langle\dot{\varepsilon}^u\rangle\frac{\partial }{\partial\varepsilon^u}\langle\frac{q}{T}f_0\mathbf{v'\cdot A}\rangle\\
    =&-\bigg\langle\frac{d\mathbf{R}}{dt}\bigg|_1\bigg\rangle\cdot\frac{\partial f_0}{\partial\mathbf{R}}\\
    &-q\mathbf{v}_g\cdot\nabla\langle\Psi\rangle\frac{f_0}{T}\\
    &+q[\mathbf{U(R)}\cdot\nabla\langle\Psi\rangle-\langle\mathbf{U}\cdot\nabla\Psi\rangle]\frac{f_0}{T}\\
    &-\frac{q}{\Omega}\bigg\langle(\nabla\Psi\times\mathbf{b})\cdot\nabla\psi_p\ \omega_0'R\bigg(\frac{B_t}{B_0}v_\parallel'+U\bigg)\bigg\rangle\frac{f_0}{T}\\
    &+S_1+S_2+S_3
\end{split}
\end{equation}
There are two terms to be added in the weight equation. For electrostatic model, $\Psi=\phi$.

The first term, 
\begin{equation}
\begin{split}
    q[\mathbf{U(R)}\cdot\nabla\langle\phi\rangle-\langle\mathbf{U}\cdot\nabla\phi\rangle]\frac{f_0}{T}
\end{split}
\end{equation}
All right, let's proceed a further step. Write
\begin{equation}
    \mathbf{U(x)=U(R) + \boldsymbol{\rho}\cdot\nabla U},
\end{equation}
then
\begin{equation}
    \begin{split}
        \mathbf{U(R)}\cdot\nabla\langle\phi\rangle-\langle\mathbf{U}\cdot\nabla\phi\rangle=-\langle\boldsymbol{\rho}\cdot\nabla \mathbf{U}\cdot\nabla\phi\rangle.
    \end{split}
\end{equation}
{\color{cyan}Remember,
\begin{equation*}
\hat{\zeta}\cdot\nabla\hat{R}=\frac{1}{R}\hat{\zeta}\ \textrm{and}\ \hat{\zeta}\cdot\nabla\hat{\zeta}=-\frac{1}{R}\hat{R}.   
\end{equation*}
These two terms results from the reconverting from the Cartesian coordinate to the cylindrical coordinate in deriving the material derivative,
see Eq. 7.106 in \cite{vc}.

\begin{equation*}
    \boldsymbol{\rho} = \rho(\mathbf{e_1}\textrm{sin}\gamma + \mathbf{e_2}\textrm{cos}\gamma),
\end{equation*}
where $\rho = v_{\perp}'/\Omega=\sqrt{2\mu B/m}/(qB/m), \mathbf{e_1}=\nabla r/|\nabla r|, \mathbf{e_2}=\mathbf{b}\times\mathbf{e_1}$ and 
$\gamma$ is the gyro angle.

Let
\begin{equation*}
    \bf{e_1}=\rm{e_{1R}}\it(r,\theta)\hat{R} + \rm{e_{1Z}}\it(r,\theta)\hat{Z}
\end{equation*}
\begin{equation*}
    \bf{e_2}=\rm{e_{2R}}\it(r,\theta)\hat{R} + \rm{e_{2Z}}\it(r,\theta)\hat{Z} + \rm{e_{2\zeta}}\it(r,\theta)\hat{\zeta},
\end{equation*} 
then we have
\begin{equation*}
    \boldsymbol{\rho}=\rho\bigg(({\rm e_{1R}sin}\gamma + {\rm e_{2R}cos}\gamma)\hat{R}
    + ({\rm e_{1Z}sin}\gamma + {\rm e_{2Z}cos}\gamma)\hat{Z} + ({\rm e_{2\zeta}cos}\gamma)\hat{\zeta}\bigg)
\end{equation*}
}
\begin{equation}
    \begin{split}
        \boldsymbol{\rho}\cdot\nabla \mathbf{U}&=\boldsymbol{\rho}\cdot\nabla(U\hat{\zeta})\\
        &=\boldsymbol{\rho}\cdot\nabla U\hat{\zeta} + U\boldsymbol{\rho}\cdot\nabla \hat{\zeta}\\
        &=\boldsymbol{\rho}\cdot\bigg(\frac{U}{R}\hat{R}-\omega'R\frac{\partial r}{\partial R}\hat{R}-\omega'R\frac{\partial r}{\partial Z}\hat{Z}\bigg)\hat{\zeta}
        -\frac{U\rho_\zeta}{R}\hat{R}\\
        &=\bigg(\frac{U\rho_R}{R} -\omega'R\frac{\partial r}{\partial R}\rho_R-\omega'R\frac{\partial r}{\partial Z}\rho_Z\bigg)\hat{\zeta}
        -\frac{U\rho_\zeta}{R}\hat{R}\\
    \end{split}
\end{equation}

\begin{equation}
    \begin{split}
        &\bigg(\frac{U\rho_R}{R} -\omega'R\frac{\partial r}{\partial R}\rho_R-\omega'R\frac{\partial r}{\partial Z}\rho_Z\bigg)\hat{\zeta}\cdot\nabla\phi\\ 
        =&\bigg(\frac{U\rho_R}{R} -\omega'R\frac{\partial r}{\partial R}\rho_R-\omega'R\frac{\partial r}{\partial Z}\rho_Z\bigg)\hat{\zeta}\cdot\frac{\partial\phi}{\partial y}\nabla y\\
        =&\bigg(-\frac{U\rho_R}{R} +\omega'R\frac{\partial r}{\partial R}\rho_R+\omega'R\frac{\partial r}{\partial Z}\rho_Z\bigg)\frac{r_0}{q_0R}\frac{\partial\phi}{\partial y}  
    \end{split}
\end{equation}

\begin{equation}
    \begin{split}
        -\frac{U\rho_\zeta}{R}\hat{R}\cdot\nabla\phi &= -\frac{U\rho_\zeta}{R}\hat{R}\cdot\bigg(\frac{\partial\phi}{\partial x}\nabla x
        + \frac{\partial\phi}{\partial y}\nabla y + \frac{\partial\phi}{\partial z}\nabla z\bigg)\\
        &=-\frac{U\rho_\zeta}{R}\bigg(\frac{\partial\phi}{\partial x}\frac{\partial r}{\partial R} 
        + \frac{\partial\phi}{\partial y}\bigg(\frac{\partial y}{\partial r}\frac{\partial r}{\partial R}
        + \frac{r_0}{q_0}\hat{q}\frac{\partial \theta}{\partial R}\bigg)
        + \frac{\partial \phi}{\partial z}q_0R_0\frac{\partial \theta}{\partial R}\bigg)
    \end{split}
\end{equation}

\begin{equation}
    \begin{split}
        &-\langle\boldsymbol{\rho}\cdot\nabla \mathbf{U}\cdot\nabla\phi\rangle=
        \bigg(\frac{U\langle\rho_R\rangle}{R} -\omega'R\frac{\partial r}{\partial R}\langle\rho_R\rangle -\omega'R\frac{\partial r}{\partial Z}\langle\rho_Z\rangle\bigg)\frac{r_0}{q_0R}\bigg\langle\frac{\partial\phi}{\partial y}\bigg\rangle\\
        &+\frac{U\langle\rho_\zeta\rangle}{R}\bigg(\bigg\langle\frac{\partial\phi}{\partial x}\bigg\rangle\frac{\partial r}{\partial R} 
        + \bigg\langle\frac{\partial\phi}{\partial y}\bigg\rangle\bigg(\frac{\partial y}{\partial r}\frac{\partial r}{\partial R}
        + \frac{r_0}{q_0}\hat{q}\frac{\partial \theta}{\partial R}\bigg)
        + \bigg\langle\frac{\partial \phi}{\partial z}\bigg\rangle q_0R_0\frac{\partial \theta}{\partial R}\bigg)
    \end{split}
\end{equation}

\begin{equation*}
    \rho_R = \rho({\rm e_{1R}sin}\gamma + {\rm e_{2R}cos}\gamma)
\end{equation*}
\begin{equation*}
    \rho_Z = \rho({\rm e_{1Z}sin}\gamma + {\rm e_{2Z}cos}\gamma)
\end{equation*}
\begin{equation*}
    \rho_\zeta = \rho{\rm e_{2\zeta}cos}\gamma
\end{equation*}
\begin{equation*}
    {\rm e_{1R}} = \frac{\partial r}{\partial R}\bigg/|\nabla r| = {\tt srbr/gr}
\end{equation*}
\begin{equation*}
    {\rm e_{1Z}} = \frac{\partial r}{\partial Z}\bigg/|\nabla r| = {\tt thbr/gr}
\end{equation*}
\begin{equation*}
    {\rm e_{2R}} = -\frac{f}{RB}{\rm e_{1Z}} 
\end{equation*}
\begin{equation*}
    {\rm e_{2Z}} = \frac{f}{RB}{\rm e_{1R}} 
\end{equation*}
\begin{equation*}
    {\rm e_{2\zeta}} = \frac{\psi_p'}{RB}\bigg[-{\rm e_{1R}}\frac{\partial r}{\partial R} - {\rm e_{1Z}}\frac{\partial r}{\partial Z}\bigg] 
\end{equation*}

The gyro average of $\rho_R, \rho_Z {\rm \ and\ } \rho_\zeta$ could be obtained by the 4-point average method.
The 4 points are at $\gamma=0,\pi/2,\pi {\rm \ and\ }3\pi/2$, respectively.

The second term,
\begin{equation}
\begin{split}
    &-\frac{q}{\Omega}\bigg\langle(\nabla\phi\times\mathbf{b})\cdot\nabla\psi_p\ \omega_0'R\bigg(\frac{B_t}{B_0}v_\parallel'+U\bigg)\bigg\rangle\frac{f_0}{T}\\
    =&-\frac{q}{\Omega}\bigg\langle(\nabla\phi\times\mathbf{b})\cdot\nabla\psi_p\bigg\rangle\omega_0'R\bigg(\frac{B_t}{B_0}v_\parallel'+U\bigg)\frac{f_0}{T}\label{t2}
\end{split}
\end{equation}
{\color{blue}
Considering
\begin{equation}
\begin{split}
    \mathbf{B}&=\nabla\psi\times\nabla(q\theta_f-\zeta)\\
    &=\frac{q_0}{r_0}\frac{d\psi}{dx}\nabla x\times\nabla y\\
    &=C(x)\nabla x\times\nabla y
\end{split}
\end{equation}
then
\begin{equation}
    \mathbf{b}=\frac{\nabla x\times\nabla y}{|\nabla x\times\nabla y|}.
\end{equation}
So we have
\begin{equation}
    \begin{split}
        \nabla\phi\times\mathbf{b}\cdot\nabla\psi_p &= \nabla\psi_p\times\nabla\phi\cdot\mathbf{b}\\
        &=\psi_p'\nabla x\times\nabla\phi\cdot\frac{\nabla x\times\nabla y}{|\nabla x\times\nabla y|}\\
        &=\frac{\psi_p'}{|\nabla x\times\nabla y|}\bigg(\frac{\partial\phi}{\partial y}\nabla x\times\nabla y
        + \frac{\partial\phi}{\partial z}\nabla x\times\nabla z\bigg)\cdot\nabla x\times\nabla y\\
        &=\psi_p'|\nabla x\times\nabla y|\frac{\partial\phi}{\partial y} + \frac{\psi_p'}{|\nabla x\times\nabla y|}
        \frac{\partial\phi}{\partial z}(\nabla x\times\nabla z)\cdot(\nabla x\times\nabla y)\label{t1}
    \end{split}
\end{equation}
\begin{equation}
    \begin{split}
        (\nabla x\times\nabla z)\cdot(\nabla x\times\nabla y)&=(\nabla x\times\nabla y)\times\nabla x\cdot\nabla z\\
        &=(|\nabla x|^2\nabla y - |\nabla x\cdot\nabla y|\nabla x)\cdot\nabla z\\
        &=|\nabla x|^2\nabla y\cdot\nabla z - |\nabla x\cdot\nabla y|\nabla x\cdot\nabla z\\
        &=|\nabla x|^2\nabla y\cdot\nabla z - q_0R_0|\nabla x\cdot\nabla y|\ |\nabla r\cdot\nabla\theta|
    \end{split}
\end{equation}
\begin{equation}
    \begin{split}
        \nabla y\cdot\nabla z &= \bigg(\frac{\partial y}{\partial r}\nabla r + \frac{r_0}{q_0}\hat{q}\nabla\theta 
        - \frac{r_0}{q_0}\nabla\zeta\bigg)\cdot q_0 R_0\nabla\theta\\
        &=q_0R_0\frac{\partial y}{\partial r}|\nabla r\cdot\nabla\theta| + r_0R_0\hat{q}|\nabla\theta|^2\label{t3}
    \end{split}
\end{equation}
}
According to Eq. \ref{t1} - \ref{t3}, Eq. \ref{t2} can be coded using existing variables.
\newpage
\bibliographystyle{unsrt}
\bibliography{refs}
\end{CJK*}
\end{document}